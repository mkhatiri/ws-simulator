\documentclass{article}
\usepackage{color}
\usepackage[utf8]{inputenc}



\usepackage[a4paper,top=3cm,bottom=2cm,left=3cm,right=3cm,marginparwidth=1.75cm]{geometry}



\title{Travail sur Work Stealing avec Fred, Denis, Nicolas}
\begin{document}
\maketitle{}
   \begin{abstract}
       Dans ce document, on va détailler les différents travaux sur le work stealing
       (ce qu'on a, ce qu'on a fait, ce qu'il faut finir, et ce qu'on va faire)
       le but de ce document est de clarifier les choses et de savoir qu'est-ce qu'on peut faire avec les données qu'on a et ce qu'on peut faire.
   \end{abstract}



   \section{Work Stealing avec Communication}

   Le défi de l'algorithme work stealing sur les plates-formes distribuées est de prendre en considération
   \textcolor{red}{le temps de communication pour transfert les tâches} après un vol réussit entre deux processeurs (voleur et victime).
   \\\\
   \textbf{Coût de communication en tant que latence:}
   le cas où on considère la latence permet de donner \textcolor{red}{le même coût pour toutes les communications}
   (pour envoyer une demande de travail, pour envoyer une réponse vide ou envoyer un ensemble de tâches).
   \\\\
   \textbf{Coût de communication en tant que Band passante:}
   Dans le cas où on considère la Band passante, le coût communication entre deux processeurs s'influence par deux facteurs :
   1-) la quantité de données transférées entre les deux processeurs,
   2-) la charge qui existe dans le canal de communication qui lie les deux processeurs concernés.

    Le cas de la band passante pose la question suivante :
    \textcolor{blue}{quelle est la topologie de communication qu'on va utiliser? :}
    (un seul canal de communication entre tous les processeurs ? un canal entre chaque deux processeurs? ou quoi ? )


   \section{Notre simulateur}
    Pour les simulations, j'ai développé avec Fred un simulateur paramétrable,
    qui peut exécuter un ensemble de taches sur un ensemble de machines suivants différents algorithmes.
    Je vais détailler dans cette partie les fonctionnalités et les limites de notre simulateur :
\\\\
    \textbf{les tâches:} le simulateur peut prendre en entrée une quantité de travail $W$,
    cette quantité peut être gérée comme un ensemble de tâches indépendantes.
    Le simulateur peut aussi prendre un seuille avec $W$ pour créer un arbre de tâches
    (les feuilles sont des tâches réelles dont la taille égale à peu pès au seuille et les nodes sont des taches de taille 0).

    Le simulateur peut aussi prendre comme entrée un fichier JSON qui contient des arbres réels avec toutes les informations sur les tâches et les précédences.
\\\\
    \textbf{les processeurs:} Le simulateur permet de créer un ensemble de processeurs dans deux topologies:
    \begin{enumerate}
        \item Un seul cluster de $p$ processeurs ou la communication égale la latence donner par l'utilisateur.
        \item Deux clusters avec le même nombre de processeurs ou la communication à l'intérieur et à l'extérieur sont configurables.
    \end{enumerate}
\textbf{les communication:} Le simulateur ne prend en compte pour le moment que la latence dans les Communication, dans les deux version un et deux clusters.
\textcolor{red}{\textbf{la version qui prend en compte la band passante neccesit une discussion pour definir l'architecture et les differents régles}}
\newpage
    \textbf{l'algorithme WS:} Le simulateur utilise le work stealing classique et plusieurs paramètres sont configurables :
        \begin{enumerate}
            \item REMOTE STEAL PROBABILITY : variable qui défini la probabilité pour volé à l'extérieur dans le cas de deux clusters.
            \item LOCAL GRANULARITY
            \item REMOTE GRANULARITY
            \item TASK THRESHOLD
            \item SIMULTANEOUSLY STEAL
        \end{enumerate}


   \section{les premier travaux : sur 1 cluster}
   Dans cette partie, on a utiliser notre simulateur,
   on analysé les résultats de simulation en changant plusieurs parammettres.
   la premier tentative a été de faire un simple fitting,
   et extraire une borne sur le temps d'execution. (cette tentative a été refuser par WAOWA conferences)

   Aprés on a bossé sur une eprouve mathématique


   
   le projet sur un seul cluster est soumiss dans TOPC journal, on attent encore une réponse.
   Un résumer est présenté a ECCO 

   \section{en cours : exp sur 2 cluster}

   \section{les idées :}


\end{document}
